\chapter{Vehicle Routing}
\label{vehicle-routing}
The final phase of our process is to determine the most efficient route for delivering the ingredients from the central depot to the delivery centers. We have modeled this problem as a Vehicle Routing Problem with Capacity constraints (VRPC). Given that there are 77 facilities, developing an optimal solution is a challenging task. As a result, we have explored three different solutions to address this issue. These include a polynomial approach, a method with subtour elimination, and a third approach that uses column generation and a custom branch and price model.

Of particular note is that while most formulations are independent of the precision of the numbers, the column generation model has a subproblem of cost $\mathbb{O}(Q^2N^2)$ where Q is the maximum capacity of the vehicles, thus, to have faster subproblem iterations the granularity of the vehicle capacity has been reduced, a similar reason is also used for the granularity of the distance. To offer better comparison between the different methodologies, the same granularities have been applied to the polynomial and subtour-elimination approaches.

\section{Polynomial model}
\label{polynomial-model}
The first method that we explored is a polynomial approach, which is the most straightforward solution. This particular formulation is a modified version of the polynomial TSP solution explained in the lectures. Despite its simplicity, this method is effective at identifying a feasible route for delivering ingredients to the various delivery centers.

\begin{align*}
    \min \quad & \sum_{i,j}{d_{i,j}x_{i,j}}\\
    \textrm{s.t.} \quad
      &\sum_{i \ne j} {x_{i,j}} = 1 \quad \forall j \in V \tag{1}\\
      &\sum_{j \ne i} {x_{i,j}} = 1 \quad \forall i \in V \tag{2}\\
      &\sum_{i \ne 0} {x_{0,i} - x_{i, 0}} = 0 \tag{3}\\
      &c_0 = 0 \tag{4}\\
      &c_j - c_i \geq w_j - M(1 - x_{i,j}) \quad \forall (i,j) \in A | i \ne j \land j \ne 0 \tag{5}\\
      &c_i \leq Q \quad \forall i \in V \tag{6}\\
      &x_{i,j} \in \{0,1\} \\
      &c_{i} \in \mathbb{N} \\
\end{align*}

Where $w_i$ is the material need of node $i$, $d_{i,j}$ is the distance from node $i$ to node $j$, $V$ is the set of nodes and $A$ is the set of arcs. The variables are: $x_{i, j}$ that is 1 if and only if the arc from node $i$ to node $j$ is selected, and $c_i$ that contains the accumulated capacity at node $i$.
The model minimizes the total distance, constraints 1, 2 and 3 are the flow constraints, while inequality 4, 5 and 6 constrain the capacity of the vehicles.

Constraint 5 activates when the arc $(i, j)$ is selected and constrains the capacity of the new node, as the inequality can be rewritten as $c_j >= c_i + w_j$. When node capacities are positive this inequality also imposes cycle elimination as $c_i$ must be strictly monotonic along the path.

\section{Subtour elimination model}
\label{subtour-elimination-model}
The second tested model is based on subtour elimination, also known as row generation. The general idea is that the model has a polynomial number of variables but an exponential number of constraints that are added incrementally during the exploration of the solution space. Usually the solution does not need to generate too many constraints, rendering this methodology usually faster than a polynomial formulation.

% TODO: multiple alignment (to align foralls)?
\begin{align*}
    \min \quad & \sum_{i,j \in A}{d_{i,j}\sum_{k \in K}{x_{i,j,k}}}\\
    \textrm{s.t.} \quad
      &x_{i,i,k} = 0 \quad \forall i \in V, k \in K \tag{1}\\
      &x_{i, 0, k} = 0 \quad \forall i \in V, k \in K \tag{2}\\
      &x_{n+1, i, k} = 0 \quad \forall i \in V, k \in K \tag{3}\\
      %
      &\sum_{j \in V,k \in K} {x_{i,j,k}} = 1 \quad \forall i \in C \tag{4}\\
      &\sum_{i \in C} {d_i\sum_{j \in V}{x_{i,j,k}}} <= Q \quad \forall k \in K \tag{5}\\
      % Flow constraints
      &\sum_{j \in V} {x_{0,j,k}} = 1 \quad \forall k \in K \tag{6}\\
      &\sum_{i \in V} {x_{i,n+1,k}} = 1 \quad \forall k \in K \tag{7}\\
      &\sum_{i \in V} {x_{i,h,k}} = \sum_{k \in V} {x_{h,j,k}} \quad \forall h \in C, k \in K \tag{8}\\
      % The exponential one
      &\sum_{(i, j) \in A} {x_{i,j,k}} \leq |S| - 1 \quad \forall S \subset V, k \in K \tag{9} \\
      &x_{i,j,k} \in \{0,1\} \\
\end{align*}

Here the nodes are a bit different than the previous model, node $0$ and $n+1$ are the depot node, and while $C$ is the set of customers ($n = |C|$), $V$ is the set of all nodes ($V = C \cup \{0, n+1\}$). The other used set is $K$ the set of usable vehicles, while $Q$, as in the other models is the maximum capacity for each vehicle. In this formulation we have only one three-dimensional variable, $x_{i,j,k}$ that is 1 if and only if vehicle $k$ uses the arc $(i,j)$.

The first three inequalities are just to remove unwanted arcs, 1 is to avoid 1-cycles, 2 and 3 remove cycles in and out of source and sink nodes. Constraint 4 allows each customer to be visited and constraint 5 limits the capacity for each vehicle. Flow constraints are enforced through inequalities 6 to 8.

The last constraint purpose is to remove cycles in each tour, but needs to be enforced for each possible cycle. To render this efficient we relax the problem and only add the constraints encountered when the problem solution contains cycles.

We adopted two additional details to increase the model performance: the first optimization is to use lazy constraints to speed up the optimization. The second one is less obvious, when a cycle is found we do not only remove that cycle but we remove the cycle also with the other $k$ (the other vehicles), this speeds up by a lot the convergence, otherwise the algorithm would try the same cycle with another vehicle.

\section{Column generation model}
\label{column-generation-model}

Column generation is a more complex approach, as such we have implemented the method described by Desrochers, Desrosiers and Solomon~\cite{desrochers1992new} simplified to remove time constraints and with a custom subproblem solution.

\begin{align*}
  \min \quad & \sum_{r \in R}{c_rx_r}\\
  \textrm{s.t.} \quad
    &\sum_{r \in R} {a_{i, r}x_r} \geq 1 \quad \forall i \in C \tag{1}\\
    &\sum_{r \in R} {x_r} = X_d \tag{2}\\
    &\sum_{r \in R} {c_rx_r} = X_c \tag{3}\\
    &x_{r} \in \{0,1\} \\
    &X_d, X_c \geq 0, integer.\\
\end{align*}


$C$ is the set of customers while $R$ is the set of all possible routes, in this model the $R$ set grows exponentially with the number of clients.

The variable $x_r$ is 1 iff the route is used, $X_d$ is the number of vehicles used and $X_c$ is the total distance traveled. Parameter $c_r$ is the total cost of route $r$ and $a_{i, r}$ counts how many times customer i is visited by route r. The system imposes integrality on the distance traveled, as such, the weights of the arcs should be integral too. To maintain precision while performing distance integralization, we defined a distance granularity parameter.

Constraint 1 requires all customers to be statisfied, while constraints 2 and 3 constrain the variables $X_c$ and $X_d$ to adhere to their definition, these are useful in the Branch and Price phase.

To resolve this problem without using an exponential number of variables, the system is first relaxed of its integrality constraints, and is then initialized with a feasible set of variables. After that the linear problem is solved and the dual variables are used to find better routes to add as variables. This process is repeated until an optimal solution for the relaxed problem is found. To then find the integral solution another method called Branch and Price is required.

\subsection{Initial routes}
There are various strategies to select initial routes, one possible solution is to select the Identity matrix, interpreted as using one vehicle to visit each customer, while this does work it leads to inefficient first steps and additional subproblem cycles.
In this implementation the initial routes are selected with a custom heuristic that explores the closest nodes until the vehicle is filled, the pseudocode is written in Algorithm~\ref{alg:initial_routes}
\begin{algorithm}[htb]
  \caption{Algorithm to find an initial solution}\label{alg:initial_routes}
  \begin{algorithmic}
    %\Input{C set of customers, $d_{i,j}$ distance of nodes, $m_i$ material required by customer i}
    \State $routes \gets \{\}$
    \State $visited \gets \{\}$
    \State $current\_node \gets 0$
    \State $current\_route \gets \{\}$
    \State $current\_capacity \gets 0$

    \While{there are still nodes to visit}
      \State $n \gets closest\_unexplored\_node(current\_node)$
      \State {Add n to visited}
      \If{$current\_capacity + m_n \geq Q$}
        \State{Close path and add it to routes}
      \EndIf
      \State {Add $n$ to $current\_route$}
      \State $current\_node \gets n$
      \State $current\_capacity \gets current\_capacity + m_n $
    \EndWhile

    \State{Close the last path and add it to routes}
  \end{algorithmic}
\end{algorithm}


\subsection{Subproblem}
The used subproblem is a variation of what was described in the original paper, we revisited it to remove the time constraints and to be more intuitive.

Given the reduced costs of each arc: $\overline{c}_{i,j} = (1 - \pi_c)c_{i,j} - \pi_i$ where $\pi_0 = \pi_d$, the subproblem requires finding negative cost routes (from the depot back to the depot) with capacity constraints traversing the graph. The graph can also contain negative loop cycles, making the problem NP hard in the strong sense.

Our solution is a simple Dynamic Programming algorithm similar to the Pulling Algorithm described in the paper, of complexity $\mathbb{O}(Q^2N^2)$
\begin{align*}
  &C_0(0) = 0 \\
  &C_j(q) = \min_{i \in V} \{G_i(q\prime) + \overline{c}_{i,j} | q\prime + q_j \leq Q\}
\end{align*}

This computes the minimal paths for each of the nodes, available in $C_i(Q) \forall i \in V$.

In addition to this the implementor will want to keep a predecessor matrix to keep track of the real computed route. This also helps in determining the 2-cycle elimination algorithm, the idea is to keep track of the best two routes for each state, one that is the best route, and the second one that is the best route using a different predecessor node.

\begin{align*}
  C_0(0) = 0 \\
  &P_j(q) = arg\,min_{i \in V} \{
    [C_i(q^\prime) + \overline{c}_{i,j} \text{if} j \neq P_i(q), C^\prime_i(q^\prime) + \overline{c}_{i,j}] |
    q^\prime + q_j \leq Q
  \} \\
  &C_j(q) = C_{P_j(q)} + \overline{c}_{i,P_j(q)} \\
  &C^\prime_j(q) = \min_{i \in V} \{
    [C_i(q^\prime) + \overline{c}_{i,j} \text{if} j \neq P_i(q), C^\prime_i(q^\prime) + \overline{c}_{i,j}] |
    q^\prime + q_j \leq Q \land i \neq P_j(q)
  \}
\end{align*}

A simple interpretation of this algorithm is that $C_i(q)$ offers the best route to arrive at node i with node capacity q, $C^\prime_i(q)$ offers the second best route and $P_i(q)$ contains the predecessor node to obtain $C_i(q)$. In a real algorithm $P^\prime_i(q)$ should also be computed to reconstruct the real paths. $C_i(q)$ selects the best paths using capacities less than $q - q_j$, also checking that the path doesn't to 2-cycles. By iterating the problem with every node at every capacity we obtain the minimal route to arrive at each node.

The subproblem is a critical component of the algorithm as it is where the model will spend most of the time, to have faster iterations we implemented the subproblem using Cython~\cite{behnel2011cython}, while all of the other project is developed using Python. This alone has achieved a 10x speedup for the solution of every subproblem.

Since the complexity of the subproblem highly depends on $Q$, the maximum capacity for each vehicle, as a speed optimization we offer a capacity granularity, by having less granularity the results are less precise but the subproblem speed decreases quadratically. All the results have a capacity granularity of 1000 Kg per customer where $Q = 10000Kg$.

\subsection{Branch and Price}
The Branch and Price algorithm is the same as described in the original algorithm, we implemented some optimizations to de-duplicate paths and other minor implementation details. The exploration direction in the algorithm is parametrized but by default a mixed exploration is selected as it quickly arrives at an integral solution, and it keeps the number of unexplored nodes comparatively low.

Developing a Branch and Price technique is quite error-prone for a student, as such we integrated a debug mode that enables exploration of the Branch and Price tree after the optimal solution has been found, keeping track of the Gurobi model and added constraints. This online debugging strategy has been a helpful way to locate and remove multiple coding mistakes.
